\documentclass[12pt]{article}

\usepackage{minted}

\title{Lab 1: Logik för dataloger}

\author{Lukas Andersson}
\date{\today}

\begin{document}

\maketitle

\section*{Uppgift 1: Unifiering}
Betrakta denna fråga till ett Prologsystem:

\begin{minted}{prolog}
?- T=f(a,Y,Z), T=f(X,X,b).
\end{minted}
Vilka bindningar presenteras som resultat?
\\
Ge en kortfattad förklaring till ditt svar!

\subsection*{Lösning}
Då T binds till \texttt{f(a,Y,Z)} och sedan till \texttt{f(X,X,b)} så måste \texttt{f(a,Y,Z)} unifieras med \texttt{f(X,X,b)}. Detta leder till att \texttt{X = a} eftersom \texttt{a} är en individkonstant och inte en variabel. Då \texttt{X = a} så måste \texttt{Y = a} och \texttt{Z = b}. Därmed är svaret att \texttt{X = a}, \texttt{Y = a} och \texttt{Z = b}.
\\
\\
Utmatningen från Prologsystemet blir:
\begin{minted}{prolog}
T = f(a, a, b),
X = a,
Y = a,
Z = b.
\end{minted}

\section*{Uppgift 2: Representation}

En lista är en representation av sekvenser där 
den tomma sekvensen representeras av symbolen \texttt{[]} 
och en sekvens bestående av tre heltal 1 2 3 
representeras av \texttt{[1,2,3]} eller i kanonisk syntax 
\begin{minted}{prolog}
'.'(1,'.'(2,'.'(3,[])))
\end{minted}

Den exakta definitionen av en lista är:

\begin{minted}{prolog}
list([]).
list([H|T]) :- list(T).
\end{minted}

Vi vill definiera ett predikat som givet en lista som 
representerar en sekvens skapar en annan lista som innehåller
alla element som förekommer i inlistan i samma ordning, men 
om ett element har förekommit tidigare i listan skall det 
inte vara med i den resulterande listan.

Till exempel:
\begin{minted}{prolog}
?- remove_duplicates([1,2,3,2,4,1,3,4], E).
\end{minted}

skall generera \texttt{E=[1,2,3,4]}.

Definiera alltså predikatet \texttt{remove\textunderscore duplicates\slash 2}!
Förklara varför man kan kalla detta predikat för en funktion!

\subsection*{Lösning}

\begin{minted}{prolog}
remove_duplicates([], []).
remove_duplicates([H|T], [H|R]) :-
    remove_duplicates(T, R),
    not(member(H, R)).
remove_duplicates([H|T], R) :-
    remove_duplicates(T, R),
    member(H, R).
\end{minted}

Detta predikat kan kallas för en funktion eftersom den är helt och hållet "ren" och funktionell då den ej har några sidoeffekter och är deterministisk, alltså att samma indata alltid ger samma utdata. \texttt{remove\textunderscore duplicates/2} beskriver alltså inte bara en relation mellan två listor utan transformerar data från en lista till en annan.

\section*{Uppgift 3: Rekursion}

Definiera predikatet \texttt{partstring/3} som givet en lista som första 
argument genererar en lista F med längden L som man finner
konsekutivt i den första listan!
Alla mōjliga svar skall kunna presenteras med hjälp av 
backtracking om man begär fram dem.

Till exempel:

\begin{minted}{prolog}
?- partstring( [ 1, 2 , 3 , 4 ], L, F).
\end{minted}

genererar exempelvis:
\texttt{F=[4]} och \texttt{L=1}
eller \texttt{F=[1,2]} och \texttt{L=2}
eller också \texttt{F=[1,2,3]} och \texttt{L=3}
eller \texttt{F=[2,3]} och \texttt{L=2} 
osv.

(Notera att t.ex. \texttt{E=[1,2,4]}, \texttt{L=3} ska inte finnas med som svar, 
då 2 och 4 inte finns konsekutivt i listan \texttt{[1,2,3,4]}.)

\end{document}